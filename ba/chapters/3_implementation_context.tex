\chapter{Implementation Context}


\section{Requirements and Constraints}

\paragraph{Accuracy might vary based on many variables}

\begin{enumerate}
    \item Object texture
    \item Distance
    \item Lighting conditions
\end{enumerate}

\paragraph{Near-real-time (once every 10s or so)}

\paragraph{Limited processing power (smartphone)}


\section{Potential pitfalls}
As the development device, Google Pixel 7, does not have a depth sensor,
the Depth API will exclusively utilize depth-from-motion techniques to derive depth information from camera images.
However, it is important to note that camera-based depth-from-motion has limitations
when it comes to detecting depth in objects with minimal texture, such as walls.
This drawback could potentially present challenges.
For instance, if the objective were to recognize objects like boxes or spheres that lack texture,
the accuracy of depth information obtained from the Depth API may not be sufficient for accurate recognition. \parencite{google_llc_arcore_doc}


\section{Hard- and Softwarestack}

The mobile application will be developed for Android and tested using a Google Pixel 7.
The implementation will be carried out in Java/Kotlin using the Android Studio integrated development environment (IDE).
The Google ARCore SDK will be used to access depth information about a scene.
The SDK is available by default -- no additional libraries are required.

Algorithms will be implemented in C\texttt{++} in the \texttt{procedural-augmented-reality} project provided by Prof. Dr. Phillipp Jenke.
This project will be integrated into the application and interfaced with Kotlin through a binding layer.

\subsection{Wrapping C\texttt{++} code in Kotlin using SWIG}
SWIG is "a program development tool that automatically generates the bindings between C/C\texttt{++}
code and common scripting languages"~\parencite{beazley_swig_1996}.


\section{Libraries and external code}
The following libraries and/or publicly available code has been used in this thesis:
\begin{itemize}
    \item The RANSAC Implementation by \citeauthor{schnabel_efficient_2007} is used for primitive detection
    and is further examined in this thesis. \parencite{schnabel_efficient_2007}
    \item \citetitle{google_llc_codelab_raw_depth} provides a reference implementation for using the ARCore Raw Depth API.
    It is used as a basis for unprojecting depth image pixels into world space in this thesis.~\parencite{google_llc_codelab_raw_depth}
\end{itemize}

\subsection{Porting the Efficient RANSAC Algorithm Library to the ARM architecture}
The code for \citetitle{schnabel_efficient_2007} by \citeauthor{schnabel_efficient_2007} is implemented in C\texttt{++}
and designed to be compiled for the x86 architecture.
To use it on an Android device, the library needs to be compiled for the ARM architecture.

To compile the library for ARM, a few adjustments to the code are necessary.
The library uses the \texttt{xmmintrin.h} header file, which is an x86-specific header file
that provides access to the Streaming SIMD Extensions (SSE) instruction set. %TODO QUOTATION NEEDED
In this case it is only used to allocate and free memory using the functions \texttt{\_mm\_malloc} and \texttt{\_mm\_free}.
A conditional compilation directive is added to include the header file only if the target architecture is x86.
If the target architecture is ARM, the standard library functions \texttt{malloc} and \texttt{free} are used instead.

Furthermore, the bundled compiler used by Android Studio (clang)
is stricter than the gcc compiler and requires some minor adjustments to the code.
Most notably, the error \texttt{explicit qualification required} is fixed by adding the
\texttt{this} keyword to all member function calls.
Other adjustments include fixes like removing obsolete keywords and changing deprecated
functions of the C++ standard library with their modern counterparts.


\section{Testing}

All algorithms implemented in the \texttt{procedural-augmented-reality} project are unit tested using
Google Test, a C\texttt{++} testing framework.
To achieve this, a CMake library target \textit{backend} is defined,
that contains all functionality of the \texttt{procedural-augmented-reality} project.
A second target that contains the tests, \textit{backend-test}, is defined,
which links against \textit{backend} and the Google Test library.