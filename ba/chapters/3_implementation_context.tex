\chapter{Implementation Context}

\section{Requirements and Constraints}

\paragraph{Accuracy might vary based on many variables}

\begin{enumerate}
    \item Object texture
    \item Distance
    \item Lighting conditions
\end{enumerate}

\paragraph{Near-real-time (once every 10s or so)}

\paragraph{Limited processing power (smartphone)}


\section{Potential pitfalls}
As the development device, Google Pixel 7, does not have a depth sensor,
the Depth API will exclusively utilize depth-from-motion techniques to derive depth information from camera images.
However, it is important to note that camera-based depth-from-motion has limitations
when it comes to detecting depth in objects with minimal texture, such as walls.
This drawback could potentially present challenges.
For instance, if the objective were to recognize objects like boxes or spheres that lack texture,
the accuracy of depth information obtained from the Depth API may not be sufficient for accurate recognition. \parencite{google_llc_arcore_doc}

\section{Hard- and Softwarestack}

The mobile application will be developed for Android and tested using a Google Pixel 7.
The implementation will be carried out in Java/Kotlin using the Android Studio integrated development environment (IDE).
The Google ARCore SDK will be used to access depth information about a scene.
The SDK is available by default -- no additional libraries are required.

Algorithms will be implemented in C\texttt{++} in the \texttt{procedural-augmented-reality} project provided by Prof. Dr. Phillipp Jenke.
This project will be integrated into the application and interfaced with Kotlin through a binding layer.

\section{Libraries and external code}
\parencite{google_llc_codelab_raw_depth} provides a reference implementation for using the ARCore Raw Depth API.
It is used as a basis for converting depth images to point clouds in this project.


