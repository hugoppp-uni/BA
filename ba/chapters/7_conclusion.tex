\chapter{Conclusion and Outlook}

This thesis introduced a system for detecting primitives in a point cloud generated from depth data obtained from the Google ARCore API.
The system utilizes a custom octree implementation to efficiently store and update the point cloud information in real time.
The RANSAC implementation by~\cite{schnabel_efficient_2007}, is employed for plane detection.
The resulting planes are then used to construct a triangle mesh representing the convex hull,
which is overlaid on the camera feed.

%The performance of the algorithms and the overall system is thoroughly discussed in chapter~\ref{ch:evaluation}.
The system demonstrates high accuracy in detecting planes and constructing meshes for highly textured objects,
such as the test cube.
However, the data quality of the Depth API with low-textured objects like walls prove to be a challenge for the system,
as the depth data for these materials contains gaps, resulting in artifacts in the detection results.
Mitigating these artifacts requires increasing the bitmap resolution, but this approach causes lower-textured materials to go undetected.
For these reasons, the system is not yet suitable for creating a full mesh-representation of a scene.

Nevertheless, the system developed in this thesis provides a solid foundation for further research and development in processing depth data from ARCore or similar APIs to create meshes from a scene.
One major limitation is missing datapoints in the data obtained through the Depth from Motion technique used by ARCore.
Therefore, evaluating the use of interpolated data from the Full Depth API and its potential for improving the results of the RANSAC algorithm would be worthwhile.
Exploring the data quality of different devices, with and without depth sensors, and its impact on the performance of
the primitive detection algorithm would also be an interesting direction for future research.

Testing different primitive detection algorithms is another area for further research.
Chapter~\ref{ch:primitive-detection-algorithms} provides an overview of various primitive detection algorithms,
while~\cite{kaiser_survey_2019} provides an extensive survey of algorithms specialized for different contexts and application.
The design of the system allows for easy integration of different algorithms without affecting other components and
the ability to collect point data with a smartphone and transfer it to another device makes it easy
to evaluate and compare different algorithms on real data collected from the Depth API\@.
As a concrete example,~\cite{kaiser_survey_2019} lists that the Hough Transform is more robust to incomplete data than RANSAC\@.
Consequently, it would be valuable to investigate whether the Hough Transform is more suitable for the depth data
obtained from the Depth API\@.

As far as expanding the system, implementing the ability to detect more complex primitives,
such as cylinders and spheres, would be a logical next step.
Addressing the issue of data quality also opens up opportunities for further research, such as automatic segmentation,
classification, and labeling of objects in a scene.
Another potential area of study is the development of a system that allows objects to be moved during a session.

%In conclusion, the system developed in this thesis provides a robust foundation for building point clouds
%and detecting primitives in data obtained from the Google ARCore Depth API.



%
%\begin{enumerate}
%    \item Exploring different datastructures other than octrees for storing and updating the point cloud data
%    \item Detection of more complex primitives, such as cylinders and spheres
%    \item Detection and classification of objects within the scene
%    \item Session Management: Implement functionality to save the extracted data, enabling users to pause and later resume the session, ensuring object identifiers are maintained across sessions
%    \item Object Persistence: Extend the system to recognize and track objects even when they are moved between sessions
%\end{enumerate}
%

